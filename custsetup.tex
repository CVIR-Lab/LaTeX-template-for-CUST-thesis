% !TeX root = ./main.tex
% 首先填写论文题目等个人信息。
\custsetup{
  title              = {长春理工大学学位论文模板示例文档\custhesisversion},
  title*             = {An example of thesis template for CUST \custhesisversion},
  author             = {CVIR},%作者中文名
  author*            = {CVIR},%作者英文名
  speciality         = {信息对抗技术},%学科专业或类别领域
  speciality*        = {Information Warfare Technology},%学科专业或类别领域英文
  supervisor         = {XXX~教授},%指导教师中文
  supervisor*        = {Prof. XXX},%指导教师英文
  degree-area*      = {Master of Engineering},%硕士类型英文,工学硕士,理学说是为Master of Science,博士学位为Doctor of Philosophy
  % date               = {2017-05-01},  % 默认为今日
  % professional-type  = {专业学位类型},
  % professional-type* = {Professional degree type},
  department         = {光电工程学院},  % 院系,本科生需要填写
  student-id         = {2102211XX},  % 学号
  %secret-level       = {秘密},     % 绝密|机密|秘密|控阅,注释本行则公开
  % secret-level*      = {Secret},  % Top secret | Highly secret | Secret
  % secret-year        = {10},      % 保密/控阅期限
  % reviewer           = true,      % 声明页显示“评审专家签名”
  %
  % 数学字体
  % math-style         = GB,  % 可选:GB, TeX, ISO
  math-font          = xits,  % 可选:stix, xits, libertinus
}
% 加载宏包

% 定理类环境宏包
\usepackage{amsthm}

% 插图
\usepackage{graphicx}

% 三线表
\usepackage{booktabs}

% 跨页表格
\usepackage{longtable}

% 算法
\usepackage[ruled,linesnumbered]{algorithm2e}

% SI 量和单位
\usepackage{siunitx}

% 参考文献使用 BibTeX + natbib 宏包
% 顺序编码制
\usepackage[sort]{natbib}
\bibliographystyle{custhesis-numerical}

% 著者-出版年制
% \usepackage{natbib}
% \bibliographystyle{custhesis-authoryear}

% 本科生参考文献的著录格式
% \usepackage[sort]{natbib}
% \bibliographystyle{custhesis-bachelor}

% 参考文献使用 BibLaTeX 宏包
% \usepackage[style=custhesis-numeric]{biblatex}
% \usepackage[bibstyle=custhesis-numeric,citestyle=custhesis-inline]{biblatex}
% \usepackage[style=custhesis-authoryear]{biblatex}
% \usepackage[style=custhesis-bachelor]{biblatex}
% 声明 BibLaTeX 的数据库
% \addbibresource{bib/cust.bib}

% 配置图片的默认目录
\graphicspath{{figures/}}

% 数学命令
\makeatletter
\newcommand\dif{%  % 微分符号
  \mathop{}\!%
  \ifcust@math@style@TeX
    d%
  \else
    \mathrm{d}%
  \fi
}
\makeatother
\newcommand\eu{{\symup{e}}}
\newcommand\iu{{\symup{i}}}

% 用于写文档的命令
\DeclareRobustCommand\cs[1]{\texttt{\char`\\#1}}
\DeclareRobustCommand\env{\texttt}
\DeclareRobustCommand\pkg{\textsf}
\DeclareRobustCommand\file{\nolinkurl}

% hyperref 宏包在最后调用
\usepackage{hyperref}
