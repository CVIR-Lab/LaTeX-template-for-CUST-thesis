% !TeX root = ../main.tex

\chapter{表、图、算法}

\section{表的绘制}
按照《研究生规范》和《本科生规范》推荐,一般常用三线表,如表~\ref{tab:exampletable}所示。
\begin{table}[htbp]
  \centering
  \caption{表号和表题在表的正上方}
  \label{tab:exampletable}
  \begin{tabular}{cl}
    \toprule
    类型   & 描述                                       \\
    \midrule
    表的编排 & 一般项目由左至右横排,数据依序竖排 \\
    表的参数 & 参数应标明量和单位等   \\
    表的引用 & 根据表的label信息,采用ref进行引用       \\
    \bottomrule
  \end{tabular}
  \note{注:表注分两种,第一种是对全表的注释,用不加阿拉伯数字排在表的下边,
    前面加“注:”;第二种是和表内的某处文字或数字相呼应的注,
    在表里面用带圈的阿拉伯数字在右上角标出,然后在表下面用同样的圈码注出来}
\end{table}

表题由表序和表名组成。表序一般按照章排表,如第1章第一个表的序号为:"表1.1",以此类推。表序与表名之间空一格,表名中不允许使用标点符号。表题后不加标点。表题置于表上,居中排列,表题与表格内容均采用宋体五号字。



\section{插图}

可能听说“\LaTeX{} 只能使用 eps 格式的图片”,甚至把 jpg 格式转为eps。事实上,这种做法已无必要,当然不赞成也不反对这种转换。但是eps每次编译时都要要调用外部工具进行解析,会导致降低编译速度。所以推荐矢量图直接使用 pdf 格式,位图使用 jpeg 或 png 格式。

pdf格式比较容易生成,adobe reader就可以,adobe illustrator也能生成。当然这些软件也都可以生成eps。
\begin{figure}[htbp]
  \centering
  \includegraphics[width=0.8\textwidth]{cust-name.pdf}
  \caption{图号、图题置于图的下方}
  \label{fig:badge}
  \note{注:图注的内容不宜放到图题中。}
\end{figure}

关于图片的并排,推荐使用较新的 \pkg{subcaption} 宏包,
不建议使用 \pkg{subfigure} 或 \pkg{subfig} 等宏包。



\section{算法环境}

模板中使用 \pkg{algorithm2e} 宏包实现算法环境。关于该宏包的具体用法,请阅读宏包的官方文档。
\begin{algorithm}[htbp]
  \SetAlgoLined
  \KwData{this text}
  \KwResult{how to write algorithm with \LaTeX2e }

  initialization\;
  \While{not at end of this document}{
    read current\;
    \eIf{understand}{
      go to next section\;
      current section becomes this one\;
    }{
      go back to the beginning of current section\;
    }
  }
  \caption{算法示例1}
  \label{algo:algorithm1}
\end{algorithm}

注意,虽然可以在论文中插入算法,但是插入大段的代码有凑字数的嫌疑,可以放在放在附录中。当然也不绝对,必要情况下有的同学选择这么做,对于这些同学,建议用 \pkg{listings} 宏包。
