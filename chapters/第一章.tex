% !TeX root = ../main.tex

\chapter{绪论或引言(章标题)}%
\section{研究背景、目标、意义(1级节标题)}
\subsection{研究背景(2级节标题)}
写明研究背景意义等内容。
\subsubsection{研究目标(3级节标题)}
\paragraph{研究目标1(4级节标题)}
\subparagraph{研究目标1.1(5级节标题)}
本模板 \pkg{custhesis} 是长春理工大学本科生和研究生学位论文的 \LaTeX{}
模板, 按照《\href{https://yjs.cust.edu.cn/yjspy/lwssjdb/67113.htm}
{研究生学位论文撰写
参考规范
}》(以下简称《研究生规范》)和
《\href{https://jwc.cust.edu.cn/gzzd/xfwj/76084.htm}
{关于印发《长春理工大学本科生
毕业设计(论文)规范化要求》的通知}》以下简称《本科生规范》)的要求编写。

3级节标题通常可以不设置,6级节标题用$\textcircled{1}$表示,通常不需要设置6层标题。由于两个《规范中》没有针对各级标题的行距,段前,段后的详细规定,因此这里设置为:
\begin{enumerate}
    \item [*]章标题,单倍行距,段前24磅,段后18磅
    \item [*]一级标题,单倍行距,段前24磅,段后6磅
    \item [*]二级标题,单倍行距,段前12磅,段后6磅
    \item [*]三级标题,单位行距,段前12磅,段后6磅
    \item [*]四到刘级标题,单位行距,段前段后均为0磅。
\end{enumerate}
后续具体数值可以再custhesis.cls文件中修改,搜索“标题”即可找到。

Changchun University of Science and Technology is a university founded in 1958 by Chinese Academy of Sciences. Through years, it has developed into a provincial key university in Jilin province with Optoelectronic technology as its outstanding feature,integrating Optics, Mechanics, Electronics, Computer and Material Science as its superiority and coordinately developing in the fields of Arts, Economics, Management and Law. It has been widely acknowledged as “the Cradle for Chinese Optical Talents”.

\section{脚注}
Changchun University of Science and Technology is a university founded in 1958 by Chinese Academy of Sciences.\footnote{Holding a leading position in China, the potential research fields cover the aspects of laser technology, optoelectronic instruments, measurement techniques, advanced manufacturing technology, optoelectronic medical instruments and nanometer manufacturing}
